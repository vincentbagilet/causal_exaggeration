The credibility revolution in economics has made causal inference methods ubiquitous. At the same time, an increasing amount of evidence has highlighted that the literature strongly favors statistically significant results. I show that these two phenomena interact in a way that can substantially worsen the reliability of published estimates: even when causal identification strategies successfully reduce bias caused by confounders, they can decrease statistical power and create another type of bias, leading to exaggerated effect sizes. This exaggeration is consequential in environmental economics, as cost-benefit analyses turn estimates into decision-making parameters for policy makers. I characterize this trade-off using a formal mathematical derivation and realistic Monte Carlo simulations replicating prevailing identification strategies. I then discuss potential avenues to address it. 