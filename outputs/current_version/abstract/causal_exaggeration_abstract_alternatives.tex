The credibility revolution in economics has made causal inference methods ubiquitous. At the same time, an increasing amount of evidence has highlighted that the literature strongly favors statistically significant results. I show that these two phenomena interact in a way that can substantially worsen the reliability of published estimates: even when causal identification strategies successfully reduce bias caused by confounders, they can decrease statistical power and create another type of bias, leading to exaggerated effect sizes. This exaggeration is consequential in environmental economics, as cost-benefit analyses turn estimates into decision-making parameters for policy makers. I characterize this trade-off using a formal mathematical derivation and realistic Monte Carlo simulations replicating prevailing identification strategies. I then discuss potential avenues to address it. 

%To avoid confounding, causal identification strategies focus on a subset of the variation in the data; the plausibly exogenous part. In this paper, I argue that this can reduce statistical power and lead estimates to exaggerate true effects sizes. Using realistic Monte Carlo simulations and a mathematical derivation, I show that for the main causal strategies, a perfectly convincing identification does not guaranty an absence of ``bias'' and that improving identification can actually pull us away from the true effect. I then discuss potential avenues to address this issue.\\
		
		
		
		%Using realistic Monte Carlo simulations and a mathematical derivation, I show that for the main causal strategies, a perfectly convincing identification does not guaranty an absence of ``bias'' and that improving identification can actually pull us away from the true effect. I then discuss potential avenues to address this issue. 
		
		
		%Quasi-experimental studies make empirical economics credible. To avoid confounding, causal identification strategies focus on a subset of the variation in the data, the plausibly exogenous part. In this paper, I argue that it can reduce statistical power and lead published estimates to exaggerate true effects sizes. Using realistic Monte Carlo simulations and a mathematical derivation, I show that %for the main causal strategies, there can be a trade-off between confounding and exaggeration: we may avoid a bias at the risk of generating another one.
		%using the main causal identification strategies to avoid a bias can also generate another one. I then discuss potential avenues to address this issue.\\~\\
